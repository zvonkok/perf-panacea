\chapter{Scheduling}
\section{Perf Sched}

Another very interesting {\em perf} command is {\tt sched}. These are predefined events
that are associated with scheduling. With {\tt sched} one can see how and when the
workload is scheduled across the CPU's, when it is woken up and led to sleep.

So e.g. to know if such a simple workload like {\tt lat_mem_rd} is CPU affine or it is
scheduled all over the CPU's, let's record a sample run.

\starttyping
perf sched record lat_mem_rd 1 2048
\stoptyping

\starttyping
# show the scheduler map
perf sched map | head -n 30

                         ,*A0                                        5382.466840 secs A0 => perf:15767
                           A0                 *.                     5382.466878 secs .  => swapper:0
                           A0                  .              *B0    5382.478644 secs B0 => rcu_sched:8
                           A0                  .              *.     5382.478647 secs
                 ,*C0      A0                  .               .     5382.478680 secs C0 => ksoftirqd/4:29
                 ,*B0      A0                  .               .     5382.478682 secs
                 ,*.       A0                  .               .     5382.478685 secs
                 ,*B0      A0                  .               .     5382.498640 secs
                 ,*.       A0                  .               .     5382.498643 secs
                   .       A0             *B0  .               .     5382.508689 secs
                   .       A0             *.   .               .     5382.508691 secs
                   .       A0             *B0  .               .     5382.528655 secs
                   .       A0             *.   .               .     5382.528658 secs
                   .       A0              .  *D0              .     5382.538673 secs D0 => kworker/11:0:65
                   .       A0              .  *.               .     5382.538678 secs
                   .       A0     *E0      .   .               .     5382.658644 secs E0 => kworker/8:0:13068
                   .       A0     *.       .   .               .     5382.658646 secs
                   .       A0      .       .  *D0              .     5382.668657 secs
                   .       A0      .       .  *.               .     5382.668663 secs
                   .       A0      .       .  *D0              .     5382.798640 secs
                   .       A0      .       .  *.               .     5382.798647 secs
                   .      *F0      .       .   .               .     5382.870995 secs F0 => multipathd:261
                   .      *A0      .       .   .               .     5382.870997 secs
                   .       A0      .       .  *D0              .     5382.928688 secs
                   .       A0      .       .  *.               .     5382.928697 secs
                   .       A0      .       .   .  *G0          .     5382.992699 secs G0 => lat_mem_rd:15768
                   .      *.       .       .   .   G0          .     5382.992705 secs
                   .      *A0      .       .   .   G0          .     5382.992751 secs
                   .      *.       .       .   .   G0          .     5382.992755 secs
                   .       .       .       .  *D0  G0          .     5383.058671 secs
\stoptyping

The first four columns represent the four CPUs in the system.  $A0$,
$B0$, $C0$ are ids for each executable that is called or scheduled
during runtime of {\tt lat_mem_rd}.

Every asterisk means a scheudling event. What is evident here in the
first 30 rows, is that {\tt lat_mem_rd} first runs on $CPU0$ and is
scheduled after some time to $CPU1$ as indicated by the *D0 where
{\tt lat_mem_rd} gets a new id.

Examining the complete trace shows that the workload is scheduled
across all CPUs and not running exclusively on one or affine to one
CPU.

Another nice features of {\tt perf sched} is to report the latencies of
the scheduling events per task. These latency can be correlated with
the timestamp from the table below with the {\tt perf sched map} output
above.

\starttyping
# show scheduling latencies
# perf sched lat
 Task                  |   Runtime ms  | Switches | Maximum delay at

 cron:808              |      0.009 ms |        1 | max at: 3500.02 s
 perf:4737             |      2.890 ms |        1 | max at: 3507.16 s
 kworker/u64:0:4543    |      0.015 ms |        5 | max at: 3488.38 s
 khungtaskd:30         |      0.004 ms |        1 | max at: 3506.07 s
 kworker/2:0:2146      |      0.054 ms |       14 | max at: 3502.20 s
 ksoftirqd/1:14        |      0.003 ms |        1 | max at: 3488.79 s
 kworker/u64:1:4624    |      0.010 ms |        3 | max at: 3485.08 s
 kworker/0:2:414       |      0.042 ms |       13 | max at: 3487.43 s
 lat_mem_rd:(20)       |  23709.517 ms |      145 | max at: 3483.96 s
 rcu_sched:7           |      0.307 ms |      126 | max at: 3504.74 s
 pkcsslotd:825         |      0.010 ms |        2 | max at: 3487.48 s
 irqbalance:867        |      0.078 ms |        2 | max at: 3493.11 s
 jbd2/dasda1-8:317     |      0.090 ms |        9 | max at: 3501.99 s
 kworker/3:2:445       |      0.007 ms |        2 | max at: 3488.39 s
 migration/3:23        |      0.000 ms |        1 | max at: 3483.45 s
 multipathd:(3)        |      0.000 ms |       30 | max at: 3490.08 s
 gmain:837             |      0.027 ms |        6 | max at: 3484.02 s
 ksoftirqd/2:19        |      0.002 ms |        1 | max at: 3488.77 s
 ksoftirqd/0:3         |      0.007 ms |        4 | max at: 3495.80 s
 kworker/1:0:2740      |      0.058 ms |       27 | max at: 3499.83 s

 TOTAL:                |  23713.129 ms |      394 |

\stoptyping

Additionally the number of context switches is reported for all task
running at the time of {\tt lat_mem_rd}.

To have even more info about the separate events call {\tt perf sched}
with the {\tt script} argument.

\starttyping
# show detailed info about scheduling
# perf sched script | head -n 30 | cut -c 1-70

            perf  1921 [000] 12208.501300:  sched:sched_wakeup: perf:1
         swapper     0 [001] 12208.501303:  sched:sched_switch: swappe
            perf  1921 [000] 12208.501318:  sched:sched_switch: perf:1
            perf  1924 [001] 12208.501326:  sched:sched_wakeup: migrat
            perf  1924 [001] 12208.501327:  sched:sched_switch: perf:1
     migration/1    13 [001] 12208.501331:  sched:sched_switch: migrat
         swapper     0 [002] 12208.501332:  sched:sched_switch: swappe
         swapper     0 [000] 12208.502371:  sched:sched_wakeup: rcu_sc
         swapper     0 [000] 12208.502372:  sched:sched_switch: swappe
       rcu_sched     7 [000] 12208.502374:  sched:sched_switch: rcu_sc
         swapper     0 [000] 12208.510368:  sched:sched_wakeup: rcu_sc
         swapper     0 [000] 12208.510370:  sched:sched_switch: swappe
       rcu_sched     7 [000] 12208.510372:  sched:sched_switch: rcu_sc
         swapper     0 [000] 12208.518367:  sched:sched_wakeup: rcu_sc
         swapper     0 [000] 12208.518369:  sched:sched_switch: swappe
       rcu_sched     7 [000] 12208.518371:  sched:sched_switch: rcu_sc
         swapper     0 [000] 12208.526366:  sched:sched_wakeup: rcu_sc
         swapper     0 [000] 12208.526368:  sched:sched_switch: swappe
       rcu_sched     7 [000] 12208.526369:  sched:sched_switch: rcu_sc
         swapper     0 [003] 12209.027500:  sched:sched_switch: swappe
      lat_mem_rd  1924 [002] 12209.027506:  sched:sched_switch: lat_me
      lat_mem_rd  1925 [003] 12209.027544:  sched:sched_wakeup: lat_me
         swapper     0 [002] 12209.027546:  sched:sched_switch: swappe
      lat_mem_rd  1924 [002] 12209.027549:  sched:sched_switch: lat_me
         swapper     0 [000] 12209.364191:  sched:sched_wakeup: multip
         swapper     0 [000] 12209.364192:  sched:sched_switch: swappe
      multipathd   418 [000] 12209.364194:  sched:sched_switch: multip
         swapper     0 [002] 12209.580448:  sched:sched_wakeup: multip
         swapper     0 [002] 12209.580449:  sched:sched_switch: swappe
      multipathd   417 [002] 12209.580451:  sched:sched_switch: multip
\stoptyping

In Linux 4.10 an additional command to {\tt perf sched} was added, namely
{\tt timehist}. It shows the scheduler latency by event. It includes the
time the task was waiting to be woken up and the scheduler latency
after wakeup to running.

\starttyping
# -M migration events
# -V CPU
# -w wakup events
# perf sched timehist -MVw | head -n 20
           time    cpu  012345678  task name            wait time  sch delay   run time
                                   [tid/pid]               (msec)     (msec)     (msec)
--------------- ------  ---------  -------------------- ---------  ---------  ---------
     291.481867 [0001]   i         <idle>                   0.000      0.000      0.000
     291.481925 [0005]       i     <idle>                   0.000      0.000      0.000
     291.481963 [0005]       s     Xorg[1884]               0.000      0.000      0.037 schedule_hrtimeout_range_clock <- schedu
     291.482019 [0001]   s         irq/31-nvidia[1980]      0.000      0.000      0.152 irq_thread <- kthread <- ret_from_fork
     291.482136 [0006]             perf[6319]                                           awakened: perf[6320]
     291.482140 [0001]   i         <idle>                   0.152      0.000      0.121
     291.482175 [0006]        s    perf[6319]               0.000      0.000      0.000 schedule_hrtimeout_range_clock <- schedu
     291.482562 [0001]             lat_mem_rd[6320]                                     awakened: kworker/u16:3[227]
     291.482564 [0006]        i    <idle>                   0.000      0.000      0.389
     291.482569 [0006]         m     kworker/u16:3[227]                                 migrated: terminator[6030] cpu 1 => 2
     291.482572 [0006]             kworker/u16:3[227]                                   awakened: terminator[6030]
     291.482574 [0006]        s    kworker/u16:3[227]       0.000      0.002      0.010 worker_thread <- kthread <- ret_from_for
     291.482576 [0002]    i        <idle>                   0.000      0.000      0.000
     291.482614 [0002]    s        terminator[6030]         0.000      0.004      0.037 schedule_hrtimeout_range_clock <- schedu
\stoptyping

For a graphical overview of scheduling events of a workload {\tt perf timechart} can
be used. First record the needed samples.

\starttyping
# generate diagram of scheduling events + cycles
# perf timechart record  lat_mem_rd 1 2048
\stoptyping

Now create a SVG out of the data.

\starttyping
perf timechart
\stoptyping

A very fast SVG viewer on a Linux system is /rsvg-view-3/.

\starttyping
# rsvg-view-3 output.svg
\stoptyping

\setupcaption [figure][location={bottom}]
\startplacefigure[title={Timechart of {\tt lat_mem_rd}, which is scheduled on all available {\sc cpus}}]
\externalfigure[sched][fullwidth]
\stopplacefigure

%\noindent
%\begin{tikzpicture}[remember picture,overlay]
%\node[anchor =north west, inner sep=0pt,outer sep=0pt] at (current page.north west) {\includegraphics[width=\paperwidth,height=0.45\paperheight]{./sched2.png}};
%\end{tikzpicture}
%#+END_LaTeX
