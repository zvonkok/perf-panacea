\chapter{Probes}
\section{Userspace Probing}

As mentioned in the \in[ssec:dsymbol] \about[ssec:dsymbol] section, {\em perf} can create
{\tt Tracepoints} at functions and source code lines. So as a first
step lets have a look which functions reside in the used binary.

\margintext{Show all functions in {\tt lat_mem_rd} for probing, debug symbols needed.}
\starttyping
# perf probe -x lat_mem_rd -F | head -n 20
Delta
Now
alarm@plt
bandwidth
base_initialize
benchmark_loads
benchmp
benchmp_child
benchmp_child_sigchld
benchmp_child_sigterm
benchmp_childid
benchmp_getstate
benchmp_interval
benchmp_parent
benchmp_sigalrm
benchmp_sigchld
benchmp_sigterm
bread
bytes
close@plt
\stoptyping

From earlier examinations one knows that the {\tt benchmark_loads} has the most
samples. Now show or list the source code of the specified function.

\margintext{List source code of {\tt benchmark_loads} function, debug symbols needed.}
\starttyping
# perf probe -x lat_mem_rd -L benchmark_loads | head -n 20
<benchmark_loads@/mnt/6ddc/benchmarks/lmbench-3.0-a9/src/lat_mem_rd.c:0>
      0  benchmark_loads(iter_t iterations, void *cookie)
      1  {
         	struct mem_state* state = (struct mem_state*)cookie;
      3  	register char **p = (char**)state->p[0];
         	register size_t i;
      5  	register size_t count = state->len / (state->line * 100) + 1;

      7  	while (iterations-- > 0) {
      8  		for (i = 0; i < count; ++i) {
      9  			HUNDRED;
         		}
         	}

     13  	use_pointer((void *)p);
     14  	state->p[0] = (char*)p;
     15  }


         void
\stoptyping

That's really nice, we see the source code with associated line
numbers and can have a glimpse on the function and its intent. One can
further dissect the overview with another feature, namely inspecting
the variables available at a specific source code line. Lets have a
look at line 9, where the macro {\tt HUNDRED} is located.

\starttyping
# list all variables at line=9 in benchmark_loads function
# perf probe -x lat_mem_rd -V benchmark_loads:9
Available variables at benchmark_loads:9
	@<benchmark_loads+86>
		 (unknown_type	cookie
		char**	p
		iter_t	iterations
		size_t	count
		size_t	i
		struct mem_state*	state
\stoptyping

There are some control variables {\tt i, count, iterations} but the interesting
variable here is {\tt p} . The {\tt lat_mem_rd} benchmarks uses pointer-chasing to step
with a specific stride through the caches and memory and {\tt p} is the beginning
of a circular linked list. Lets create a {\em Tracepoint} and gather the
address of {\tt p}.

\starttyping
# create probe for variable p
# perf probe -x lat_mem_rd 'bl9=benchmark_loads:9 addr=p' 2>&1
Added new event:
  probe_lat:bl9        (on benchmark_loads:9 in lat_mem_rd with addr=p)

You can now use it in all perf tools, such as:

	perf record -e probe_lat:bl9 -aR sleep 1

\stoptyping

This new probe can now be used in every perf tool as any other
event. One can count, record, ... this new event. Just for sanity
checking list the new event.

\starttyping
# check for event
# perf list bl9
probe_lat:bl9                                      [Tracepoint event]
\stoptyping

Lets try to count how many times we reach this {\em Tracepoint}.

\starttyping
# use the event either in stat or record
# perf stat -e probe_lat:bl9 lat_mem_rd 1 2048 2>&1

"stride=2048
0.00195 7.208
0.00293 7.179

    ...

 Performance counter stats for 'lat_mem_rd 1 2048':

        28,802,191      probe_lat:bl9

      23.484057250 seconds time elapsed

\stoptyping

To display the value of {\tt p}, which was gathered in the {\em Tracepoint},
the samples have to be recorded.

\starttyping
# record the events so we can examine the value of p variable
# perf record -e probe_lat:bl9 lat_mem_rd 1 2048 2>&1
"stride=2048
0.00195 8.615
0.00293 8.550
0.00391 8.365
0.00586 8.355
     ...

[ perf record: Woken up 2768 times to write data ]
Warning:
Processed 35165656 events and lost 109 chunks!

Check IO/CPU overload!

[ perf record: Captured and wrote 1912.943 MB perf.data (23951777 samples) ]
\stoptyping

Now just call {\tt perf script} to have an console output of the gathered
samples. What is evident here is that {\tt lat_mem_rd} starts always from
the same address the pointer-chasing, the variability lies in the
iteration count and stride and buffer size.


\starttyping
# show the value of p
# perf script
      lat_mem_rd  5846 [000]  5210.256441: probe_lat:bl9: (2aa01f81d7e) addr=0x3ff9affe000
      lat_mem_rd  5846 [000]  5210.256447: probe_lat:bl9: (2aa01f81d7e) addr=0x3ff9affe000
      lat_mem_rd  5846 [000]  5210.256449: probe_lat:bl9: (2aa01f81d7e) addr=0x3ff9affe000
      lat_mem_rd  5846 [000]  5210.256450: probe_lat:bl9: (2aa01f81d7e) addr=0x3ff9affe000
      lat_mem_rd  5846 [000]  5210.256451: probe_lat:bl9: (2aa01f81d7e) addr=0x3ff9affe000
      lat_mem_rd  5846 [000]  5210.256452: probe_lat:bl9: (2aa01f81d7e) addr=0x3ff9affe000
      lat_mem_rd  5846 [000]  5210.256452: probe_lat:bl9: (2aa01f81d7e) addr=0x3ff9affe000
      lat_mem_rd  5846 [000]  5210.256453: probe_lat:bl9: (2aa01f81d7e) addr=0x3ff9affe000
      lat_mem_rd  5846 [000]  5210.256454: probe_lat:bl9: (2aa01f81d7e) addr=0x3ff9affe000
      lat_mem_rd  5846 [000]  5210.256455: probe_lat:bl9: (2aa01f81d7e) addr=0x3ff9affe000
      lat_mem_rd  5846 [000]  5210.256457: probe_lat:bl9: (2aa01f81d7e) addr=0x3ff9affe000
      lat_mem_rd  5846 [000]  5210.256457: probe_lat:bl9: (2aa01f81d7e) addr=0x3ff9affe000
      lat_mem_rd  5846 [000]  5210.256458: probe_lat:bl9: (2aa01f81d7e) addr=0x3ff9affe000
      lat_mem_rd  5846 [000]  5210.256459: probe_lat:bl9: (2aa01f81d7e) addr=0x3ff9affe000
      lat_mem_rd  5846 [000]  5210.256460: probe_lat:bl9: (2aa01f81d7e) addr=0x3ff9affe000
      lat_mem_rd  5846 [000]  5210.256461: probe_lat:bl9: (2aa01f81d7e) addr=0x3ff9affe000
      lat_mem_rd  5846 [000]  5210.256462: probe_lat:bl9: (2aa01f81d7e) addr=0x3ff9affe000
      lat_mem_rd  5846 [000]  5210.256463: probe_lat:bl9: (2aa01f81d7e) addr=0x3ff9affe000
      lat_mem_rd  5846 [000]  5210.256463: probe_lat:bl9: (2aa01f81d7e) addr=0x3ff9affe000
      lat_mem_rd  5846 [000]  5210.256464: probe_lat:bl9: (2aa01f81d7e) addr=0x3ff9affe000
\stoptyping


\subsection{Backlog}
\starttyping
 perf probe -x ./lat_mem_rd -L benchmark_loads
 perf probe -x ./lat_mem_rd 'bm_load=benchmark_loads'

 perf probe -x ./lat_mem_rd -V benchmark_loads:9
 perf probe -x ./lat_mem_rd  'bl5=benchmark_loads:9 addr=p'

 perf stat -I 1000 -e probe_lat:bm_load ./lat_mem_rd 1 2048
 perf record -e probe_lat:mem_load ./lat_mem_rd 1 1024

 perf script
\stoptyping


\subsection[ssub:heatmap]{Heatmap}

In this section {\em probes}\sidenote{The probes used here are static
  tracepoints that are already available no need to create new one}
will be used as timestamp anchor points for latency measurement of the
{\em fio} benchmark. For latency measurements one needs two timestamps
where the difference between them is the latency.

Since {\em fio} exercises the disk, the two {\em probes} used here will
be {\tt block:block_rq_issue} and {\tt block:block_rq_complete}.
But first create a {\em fio} testcase.

\margintext{The {\em fio} benchmark will be doing a random read on a 4 Gb file.}
\starttyping
cat > random-read-test.fio << EOF
[random-read]
rw=randread
size=4g
directory=/tmp
EOF
\stoptyping

Next, use {\em perf} and sample the benchmark with the two {\em probes}
selected above.
\starttyping
# perf record -e block:block_rq_issue -e block:block_rq_complete -a -- fio random-read-test.fio
\stoptyping

Lets have a look at the captured data. Look for the timestamps and the events
just captured. This will be used in the next step to calculate the latencies
and feed them to the charting software.

\margintext{The {\tt --ns} switch is used to display the time using 9 decimal places. The 4th column
shows the timestamps and the 5th the corresponding event.}
\starttyping
# perf script --ns
            perf 12505 [003]  3108.108454:    block:block_rq_issue: 94,0 RM 4096 () 58789608 + 8 [perf]
         swapper     0 [003]  3108.108585: block:block_rq_complete: 94,0 RM () 58789608 + 8 [0]
            perf 12505 [003]  3108.108595:    block:block_rq_issue: 94,0 RM 4096 () 58789568 + 8 [perf]
         swapper     0 [003]  3108.108723: block:block_rq_complete: 94,0 RM () 58789568 + 8 [0]
            perf 12505 [003]  3108.108740:    block:block_rq_issue: 94,0 RA 4096 () 33569728 + 8 [perf]
            perf 12505 [003]  3108.108742:    block:block_rq_issue: 94,0 RA 4096 () 33569736 + 8 [perf]
            perf 12505 [003]  3108.108744:    block:block_rq_issue: 94,0 RA 4096 () 33569744 + 8 [perf]
            perf 12505 [003]  3108.108746:    block:block_rq_issue: 94,0 RA 4096 () 33569752 + 8 [perf]
            perf 12505 [003]  3108.108748:    block:block_rq_issue: 94,0 RA 4096 () 33569760 + 8 [perf]
         swapper     0 [003]  3108.108890: block:block_rq_complete: 94,0 RA () 33569728 + 8 [0]

\stoptyping

With a bit of {\tt awk} magic the events and the corresponding timestamps
are extracted and the latency is calculated.

\starttyping
perf script --ns | awk '{ gsub(/:/, "") } $5 ~ /issue/ { ts[$6, $10] = $4 }
    $5 ~ /complete/ { if (l = ts[$6, $9]) { printf "%.f %.f\n", $4 * 1000000,
    ($4 - l) * 1000000; ts[$6, $10] = 0 } }' > out.lat_us
\stoptyping


Now to visualize the latencies one needs additional software that is
available on {\em github}.

\starttyping
git clone https://github.com/brendangregg/HeatMap.git
\stoptyping

The last step is to use {\tt trace2heatmap.pl} to generate an interactive
\infull{SVG} (\SVG\) of the latencies.

\starttyping
trace2heatmap.pl --unitstime=us  --maxlat=2000 --unitslabel=us \
--grid --title "fio Latency (DASD)" out.lat_us > heatmap.svg
\stoptyping

Use your preferred \SVG\ renderer and have a look at the created
figure. The more red a sqaure is the more latencies were recorded.
The is a tri modal distribution of latencies that corespond to (1)
the page cache (2) to the disk cache and last the request that
went to disk.

\setupcaption [figure][location={bottom}]
\startplacefigure[title={Latencies visualized with a heatmap}]
\externalfigure[heatmap][fullwidth]
\stopplacefigure


\section{Kernel Probing}

The nice thing about perf is that it makes no differentiation between
userspace or kernelspace. What one was doing in the section before in
userspace can be done similarly for the {\em Kernel}. The focus in this
section will be on the {\tt ping} command. Thats why the examination will
be filtered from beginning on {\tt icmp_rcv}.


List the functions of the current {\em Kernel} and grep for {\tt icmp_rcv}.

\starttyping
#+BEGIN_SRC bash  :dir /sshx:root@s311lp09:/ :results value code :exports both :cache yes
# list function for probing here icmp_rcv
perf probe -k /usr/lib/debug/boot/vmlinux-4.8.0-34-generic  -F  | grep icmp_rcv 2>&1
icmp_rcv
\stoptyping

Create a probe on {\tt icmp_rcv}, this time the purpose of the probe is to count
how many times {\tt icmp_rcv} was called.

\starttyping
# create a probe on icmp_rcv
perf probe -k /usr/lib/debug/boot/vmlinux-4.8.0-34-generic   icmp_rcv 2>&1
\stoptyping

\starttyping
TODO: perf stat ping
\stoptyping

Record the samples and ping {\tt pserver1} six times.

\starttyping
# recored new event probe:icmp_rcv
# perf record -e probe:icmp_rcv -aR ping -c6 pserver1.boeblingen.de.ibm.com
PING pserver1.boeblingen.de.ibm.com (9.152.140.6) 56(84) bytes of data.
64 bytes from pserver1.boeblingen.de.ibm.com (9.152.140.6): icmp_seq=1 ttl=64 time=0.547 ms
64 bytes from pserver1.boeblingen.de.ibm.com (9.152.140.6): icmp_seq=2 ttl=64 time=0.144 ms
64 bytes from pserver1.boeblingen.de.ibm.com (9.152.140.6): icmp_seq=3 ttl=64 time=0.143 ms
64 bytes from pserver1.boeblingen.de.ibm.com (9.152.140.6): icmp_seq=4 ttl=64 time=0.140 ms
64 bytes from pserver1.boeblingen.de.ibm.com (9.152.140.6): icmp_seq=5 ttl=64 time=0.141 ms
64 bytes from pserver1.boeblingen.de.ibm.com (9.152.140.6): icmp_seq=6 ttl=64 time=0.146 ms

--- pserver1.boeblingen.de.ibm.com ping statistics ---
6 packets transmitted, 6 received, 0% packet loss, time 5095ms
rtt min/avg/max/mdev = 0.140/0.210/0.547/0.150 ms
\stoptyping


With the present {\tt perf.data} and the {\tt report} tool one can easily e.g.
find out in which source file the function {\tt icmp_rcv} is implemented.

\starttyping
# show the source file
#perf report -s srcfile --stdio 2>&1 --stdio-color=never
# To display the perf.data header info, please use --header/--header-only options.
#
#
# Total Lost Samples: 0
#
# Samples: 6  of event 'probe:icmp_rcv'
# Event count (approx.): 6
#
# Overhead  Source File
# ........  ...........
#
  100.00% icmp.c


\stoptyping

Additionally one can gather the source line of the file reported above
where the {\em Tracepoint} is located.

\starttyping
# show the source line
# perf report -s srcline --stdio 2>&1 --stdio-color=never
# To display the perf.data header info, please use --header/--header-only options.
#
#
# Total Lost Samples: 0
#
# Samples: 6  of event 'probe:icmp_rcv'
# Event count (approx.): 6
#
# Overhead  Source:Line
# ........  ...........
#
  100.00% icmp.c:973


\stoptyping

\starttyping
# vi ../6ddc/kernel/linux/net/ipv4/icmp.c 973
\stoptyping
