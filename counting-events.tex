\chapter{Counting events}
\section{perf stat}
The {\em perf} tool has two modus operandi. The (1) is counting events
and the (2) is recording (sampling) events. The focus in this section
will be on counting various events.

A very simple example of counting events is to call /perf stat/ on ls.
It shows predefined events that are associated with the /stat/ command.
As for every {\em perf} command, it can be selected which events to take
under consideration.

\starttyping
# perf stat ls
A    bin	etc    lost+found  opt		  proc	       sbin  usr
A.c  boot	home   media	   output.svg	  provisioner  srv   var
app  chroot.sh	lib    mnt	   perf.data	  root	       sys
B    dev	lib64  op	   perf.data.old  run	       tmp

 Performance counter stats for 'ls':

          0.425128      task-clock (msec)         #    0.753 CPUs utilized
                 0      context-switches          #    0.000 K/sec
                 0      cpu-migrations            #    0.000 K/sec
                94      page-faults               #    0.221 M/sec
         2,070,136      cycles                    #    4.869 GHz
         1,284,151      instructions              #    0.62  insn per cycle
   <not supported>      branches
   <not supported>      branch-misses

       0.000564777 seconds time elapsed

\stoptyping

To count the number of the raw event CPU_CYCLES , one can call /perf
stat/ with the /-e/,/--event/ parameter and as an argument the /r0/
raw event.

\starttyping
# perf stat -e r0 ls
A    bin	etc    lost+found  opt		  proc	       sbin  usr
A.c  boot	home   media	   output.svg	  provisioner  srv   var
app  chroot.sh	lib    mnt	   perf.data	  root	       sys
B    dev	lib64  op	   perf.data.old  run	       tmp

 Performance counter stats for 'ls':

         1,951,637      r0

       0.000531562 seconds time elapsed

\stoptyping

To have {\em perf} to print the stats every second for example, than put
the time in [ms] after /-I,--interval-print/.

\starttyping
# perf stat -I 1000 -e r0 lat_mem_rd 1 2048 2>&1
"stride=2048
#           time             counts unit events
     1.000112465      4,993,322,683      r0
0.00195 0.793
     2.000195215      4,997,804,011      r0
0.00293 0.793
     3.000270278      4,997,843,664      r0
     4.000333966      4,997,740,847      r0
0.00391 0.793
     4.498059029      2,487,320,780      r0
     4.498089872            158,701      r0
\stoptyping

Additionally to intervall printing a nice feature is to show how the
events are distributed across cores. Append /--pre-core/ to the /perf
stat/ command.

\starttyping
# perf stat -I 1000 -a --per-core -e r0 md5sum /boot/vmlinuz 2>&1
2d177b173d7e9ff0f00dc990708bf6fc  /boot/vmlinuz
#           time core         cpus             counts unit events
     0.005954344 S3-C0           1         29,545,117      r0
     0.005954344 S3-C1           1             37,632      r0
     0.005954344 S3-C2           1             63,045      r0
     0.005954344 S3-C3           1             97,929      r0
\stoptyping

\subsection{Counting Backlog}

\starttyping

 perf stat -B ./lat_mem_rd 1 2048
 perf stat -e task-clock,cpu-clock ./lat_mem_rd 1 2048

 perf stat -e cache-misses,L1-dcache-loads,mem-loads ./lat_mem_rd 1 2048
 perf stat -e cache-misses ./lat_mem_rd 1 2048

 # task-clock is based only on the time spent on the profiled task,
 # so that doesn't count time spent on other tasks, it has a per thread granularity

 # print out every second

 perf stat -I 1000 -e mem-loads ./lat_mem_rd 1 2048
 perf stat -I 1000 -e L1-dcache-loads ./lat_mem_rd 1 2048
 perf stat -I 1000 -e L1-dcache-loads,cache-misses ./lat_mem_rd 1 2048

 # per-core per-cpu ..

 perf stat -I 1000 -e instructions --per-core lat_mem_rd 1 2048
\stoptyping
