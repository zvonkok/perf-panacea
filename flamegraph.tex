% -*- mode:   context; -*-
\chapter[fgraph]{Flamegraph}
\section{Call-Stacks}

In \in[framep] \about[framep] it was described what a stack is and how
it works.  In this section one will use {\em perf} to capture or
sample these {\em call-stacks} and visualize the in a very specific
way, namely in a {\em flamegraph}.

But first lets print out some example call-stacks from a running
application and deduce how to interpret a {\em flamegraph}. For this
the {\tt eu-stack}\sidenote{For Java use {\tt jstacks}} tool will be
used for examination. It shows call-stacks from an application
currently running.

The application used here is {\tt pi} from the {\em gmpbench} suite. It
calculates the number $\pi$ up to the, as an argument given, decimal places.

\margintext{Calculate 1000 decimal places for the number $\pi$ using
  Chudnovsky's algorithm and show info about the \ELF\ auxiliary
  vector. \ELF\ auxiliary vectors are a mechanism to transfer certain
  kernel level information to the user processes}
\starttyping
#  LD_SHOW_AUXV=1 ./pi 1000
AT_SYSINFO_EHDR: 0x3ff8e6fe000
AT_HWCAP:    esan3 zarch stfle msa ldisp eimm dfp edat etf3eh highgprs te vx
AT_PAGESZ:       4096          /* System page size                  */
AT_CLKTCK:       100           /* Program headers for program       */
AT_PHDR:         0x2aa0cc80040 /* Frequency of times()              */
AT_PHENT:        56            /* Size of program header entry      */
AT_PHNUM:        9             /* Size of program header entry      */
AT_BASE:         0x3ff8e680000 /* Size of program header entry      */
AT_FLAGS:        0x0           /* Flags                             */
AT_ENTRY:        0x2aa0cc81c58 /* Flags                             */
AT_UID:          0             /* Real uid                          */
AT_EUID:         0             /* Effective uid                     */
AT_GID:          0             /* Real gid                          */
AT_EGID:         0             /* Effective gid                     */
AT_SECURE:       0             /* Treat executable securely         */
AT_RANDOM:       0x3ffca77e3bc /* Address containing a random value */
AT_EXECFN:       ./pi          /* Pathname used to execute program  */
AT_PLATFORM:     z13           /* Hardware platform                 */
\stoptyping
%
During the runtime of {\tt pi} one can call {\tt eu-stack -p `pidof
  pi`} periodically e.g. every 1/2 seconds to simulate the call-stack
sampling of {\tt perf record -F2 } with a frequency of {\em 2 Hz}. This way
one can manually collect 2 call-stacks per second.

Here are 3 call-stack samples from the beginning of {\tt pi}.

\margintext{The addresses for {\tt build_sieve} can be and are different, because this was the active
  stack frame, when sampled, and the address is the {\em instruction pointer} at that time. To know
where in the code one was exactly when sampled, {\tt addr2line} can be used}

\starttyping
PID 18770 - process                      PID 18770 - process                       PID 18770 - process
TID 18770:                               TID 18770:                                TID 18770:
#0  0x0000000000403bcc build_sieve        #0  0x0000000000403bc3 build_sieve        #0  0x0000000000403bff build_sieve
#1  0x0000000000403d59 picomp             #1  0x0000000000403d59 picomp             #1  0x0000000000403d59 picomp
#2  0x00000000004007db main               #2  0x00000000004007db main               #2  0x00000000004007db main
#3  0x000000000042cbe6 generic_start_main #3  0x000000000042cbe6 generic_start_main #3  0x000000000042cbe6 generic_start_main
#4  0x000000000042ce6e __libc_start_main  #4  0x000000000042ce6e __libc_start_main  #4  0x000000000042ce6e __libc_start_main
#5  0x0000000000400dea _start             #5  0x0000000000400dea _start             #5  0x0000000000400dea _start

# Use addr2line to find out at which source code line the instruction pointer pointed
# to during execution when pi was sampled

addr2line -a 0x0000000000403bcc -e ./pi   addr2line -a 0x0000000000403bc3 -e ./pi   addr2line  -a 0x0000000000403bff -e ./pi
0x0000000000403bcc                        0x0000000000403bc3                        0x0000000000403bff
gmpbench-0.2/pi.c:485                     gmpbench-0.2/pi.c:484                     gmpbench-0.2/pi.c:488

485:   s[i/2].fac = i;                    484:  if (s[i/2].fac == 0) {              488:for (j=i*i, k=i/2; j<=n; j+=i+i, k++) {

\stoptyping

Now to visualize the call-stacks one needs additional software that is
available on {\em github}.

\starttyping
git clone https://github.com/brendangregg/FlameGraph.git
\stoptyping

There are several tools to convert {\em call-stacks} from different
source ({\em perf,dtrace,...}) to a format that {\tt flamegraphs.pl}
understands. First it will be done manually and then shown how to
automate these steps with {\em perf}.

The input file for {\tt flamegraphs.pl} is as follows, each line
defines exactly one unique {\em call-stack} with the count of it.
So e.g. in the case above the {\em folded} {\em call-stack} file
should look like this.

\starttyping
# cat pi.man.folded
_start;__libc_start_main;generic_start_main;main;picomp;build_sieve 3
\stoptyping

This file can now be used to generate the {\em flamegraph}.

\starttyping
cat pi.man.folded > flamegraph.pl > pi.man.svg
\stoptyping

\setupcaption [figure][location={bottom}]
\startplacefigure[title={Flamegraph of 3 {\em call-stack} samples}]
\externalfigure[pi.man.svg][textwidth]
\stopplacefigure

\section{Perf Flamegraph}

The previous chapter showed how one can generate a {\em flamegraph} manually. Now
let's see how to automate or generate a {\em flamegraph} with {\em perf}.

First, sample the workload with the {\tt --call-graph=dwarf} switch.

\margintext{Limit the frequency of sampling to your needs, otherwise {\em perf} could
slow down your system to uselessness, there is a lot of \CPU\ and memory load. }
\starttyping
# perf record -F99 --call-graph=dwarf -- ./pi 1000
\stoptyping

Combine all the steps explained above to one line and generate the {\em flamegraph}.

\starttyping
# perf script | stackcollapse-perf.pl | flamegraph.pl > pi.svg
\stoptyping


\setupcaption [figure][location={bottom}]
\startplacefigure[title={Complete Flamegraph of {\tt pi}}]
\externalfigure[pi.svg][fullwidth]
\stopplacefigure

The generated \SVG\ is a interactive graphic, where one can search for
a symbol or just activate a specific {\em call-stack} by clicking on
it. The broader a bar is the more samples where gathered for this {\em
  call-stack} and hence more time was spent in it. Beware that {\em
  perf} was in sampling mode, which means, that {\em call-stack} that
were really small in time, can but not necesserialy have to be
sampled. If {\em call-stacks} are missing try to increase the sampling
frequency.


\section{Differential Flamegraphs}
Differential flame graphs visualize the differences between two
performance profiles. It can be used e.g. for regression runs or to
identify possible performance enhancement like compiler switches,
tuning parameters and so on. Let's try to examine how the different
optimization levels affect performance of the workload {\tt
  pi}. Therefore the workload will be compiled with {\tt -O1, -O2,
  -O3} and the corresponding {\em flamegraphs} for each run will be
generated. The last step is to generate the {\em differential
  flamegraphs} to see the differences.

First generate the workloads.
\starttyping
# gcc -O3 -g pi.c -o pi.3 -static -lgmp -lm
# gcc -O2 -g pi.c -o pi.2 -static -lgmp -lm
# gcc -O1 -g pi.c -o pi.1 -static -lgmp -lm
\stoptyping

Now for each optimization level sample the {\em call-stacks}

\starttyping
# perf record -F99 --call-graph=dwarf -o pi.1.data -- ./pi.1 1000
# perf record -F99 --call-graph=dwarf -o pi.2.data -- ./pi.2 1000
# perf record -F99 --call-graph=dwarf -o pi.3.data -- ./pi.3 1000
\stoptyping

The next step is to generate the folded stacks.

\starttyping
# perf script -i pi.1.data | stackcollapse-perf.pl > pi.1.folded
# perf script -i pi.2.data | stackcollapse-perf.pl > pi.2.folded
# perf script -i pi.3.data | stackcollapse-perf.pl > pi.3.folded
\stoptyping

And as a last step the {\em differential flamegraphs} can be created.
\starttyping
difffolded.pl -n pi.1.folded pi.2.folded | flamegraph.pl --negate > pi.diff12.svg
difffolded.pl -n pi.2.folded pi.3.folded | flamegraph.pl --negate > pi.diff23.svg
\stoptyping


\placefigure
  [here]
  [fig:fgdiff]
  {Differential Flamegraphs between compiler optimization levels}
  {\startcombination[1*2]
    {\externalfigure[pi.diff12.svg][textwidth]}
    {\externalfigure[pi.diff23.svg][textwidth]}
  \stopcombination}

Functions that are performing better are coloured blue and functions that
perform worse are coloured red. The intensity shows how much better or worse
the functions get in terms of performance.

\section{Java Flamegraph}

The nice thing about {\em perf} with {\em Java} is that it's possible
to not only sample {\em Java} like {\em Jinsight} but it can also
sample the \JVM\ with the {\em garbage collector}, userspace and the
kernel. The complete stack can be examined and analyzed.

Since {\em Java} has a deep call hierarchy one should increase the
maximum stack-frame count so that the complete {\em call-stack} is
recorded.

\starttyping
# sysctl -w kernel.perf_event_max_stack=512
\stoptyping

For the \infull{JIT} (\JIT\) compiled methods {\em perf} needs
additional information to map the samples correctly to the
symbols. {\em Perf} has a nifty mechanism for finding missing mappings
(from instruction addresses to symbols/methods). When {\tt perf
  report} encounters missing mappings it searches for maps in {\tt
  /tmp/perf-<pid>.map}. This allows runtimes that generate code on the
fly to supply dynamic symbol mappings to be used with {\em perf}.
That is the file where the next tool will create mappings for
\JIT\-compiled methods.

\starttyping
# git clone  https://github.com/jvm-profiling-tools/perf-map-agent.git
cd perf-map-agent
cmake .
make
\stoptyping

There is another issue with {\em Java} namely the {\em frame-pointer}.
The \JVM\ compiler is ommiting the {\em frame-pointer} and not generating
\DWARF\ debuginfo to unwind the stack. Luckily since \JDK\ Version 8 there
is a new option to preserve the {\em frame-pointer}. Beware that preserving
the {\em frame-pointer} the performance can degrade by up to 2\%. A prerequisite
for sampling {\em Java call-stacks} is the following option.
\margintext{This option is not (yet) available in the \IBM\ \JDK\ }
\starttyping
-XX:+PreserveFramePointer
\stoptyping

For the next steps one will use the workload {\em SPECjbb2005} with the modified
{\em Java} command line. To record and get the right maps for the workload
the scripts in {\tt perf-map-agent/bin} can be used for easy of use. But first
export  {\tt JAVA_HOME} to the correct \JDK\ .

\starttyping
export JAVA_HOME=/usr/lib/jvm/java-1.9.0-openjdk-amd64
\stoptyping

First start the workload.
\starttyping
./SPECjbb2005Run 1073741824  /usr/lib/jvm/java-1.9.0-openjdk-amd64/bin  4096m   4096m   jitc  0   0 1  ""   0  0 "keep"
\stoptyping


Now collect the {\em call-stacks}.

\starttyping
# ./perf-java-record-stack 10313
+++ readlink -f ./perf-java-record-stack
++ dirname /home/zkosic/git/perf-map-agent/bin/perf-java-record-stack
+ PERF_MAP_DIR=/home/zkosic/git/perf-map-agent/bin/..
+ PID=10313
+ '[' -z '' ']'
+ PERF_JAVA_TMP=/tmp
+ '[' -z '' ']'
+ PERF_RECORD_SECONDS=15
+ '[' -z '' ']'
+ PERF_RECORD_FREQ=99
+ '[' -z '' ']'
+ PERF_DATA_FILE=/tmp/perf-10313.data
+ echo 'Recording events for 15 seconds (adapt by setting PERF_RECORD_SECONDS)'
Recording events for 15 seconds (adapt by setting PERF_RECORD_SECONDS)
+ sudo perf record -F 99 -o /tmp/perf-10313.data -g -p 10313 -- sleep 15
[ perf record: Woken up 1 times to write data ]
[ perf record: Captured and wrote 0.681 MB /tmp/perf-10313.data (3555 samples) ]
\stoptyping


\starttyping
# perf script -i /tmp/perf-10313.data | stackcollapse-perf.pl --all | flamegraph.pl --color=java --hash > jbb.jsvg
\stoptyping



\definefloat[bleedfigure][figure]
\setupfloat
[bleedfigure]
[
  leftmargindistance=0mm,
  rightmargindistance=0mm,
default={backspace}]



%\setupcaption [figure][location={bottom}]
%\placebleedfigure{Flamegraph of {\em Java}, {\em JDK} and Kernel}{\externalfigure[jbb.svg][paperwidth]}

\setupcaption [figure][location={bottom}]
\startplacefigure[title={Flamegraph of {\em Java}, {\em JDK} and Kernel}]
\externalfigure[jbb.svg][fullwidth]
\stopplacefigure


The {\em green} colour represents {\em Java} functions the {\em yellowish} are functions
from the {\em JDK} and last but not least the {\em red} ones are from the {\em Kernel}. {\em}
Java code is inlined at a high degree and for that purpose one can supply {\tt unfoldall} as an
option to see even the inlined functions.

\starttyping
# export PERF_MAP_OPTIONS=unfoldall
# perf-map-agent/bin# ./perf-java-record-stack 11868
+++ readlink -f ./perf-java-record-stack
++ dirname /home/zkosic/git/perf-map-agent/bin/perf-java-record-stack
+ PERF_MAP_DIR=/home/zkosic/git/perf-map-agent/bin/..
+ PID=11868
+ '[' -z '' ']'
+ PERF_JAVA_TMP=/tmp
+ '[' -z '' ']'
+ PERF_RECORD_SECONDS=15
+ '[' -z '' ']'
+ PERF_RECORD_FREQ=99
+ '[' -z '' ']'
+ PERF_DATA_FILE=/tmp/perf-11868.data
+ echo 'Recording events for 15 seconds (adapt by setting PERF_RECORD_SECONDS)'
Recording events for 15 seconds (adapt by setting PERF_RECORD_SECONDS)
+ sudo perf record -F 99 -o /tmp/perf-11868.data -g -p 11868 -- sleep 15
[ perf record: Woken up 1 times to write data ]
[ perf record: Captured and wrote 0.653 MB /tmp/perf-11868.data (3414 samples) ]

\stoptyping


The result is stunning, additional to the visualization of the  complete  execution
stack one has the inlined {\em Java} functions. Newer {\em Perf} version can handle
even inlined {\sc c/c++} code.

\setupcaption [figure][location={bottom}]
\startplacefigure[title={Flamegraph of {\em Java} (with inlined functions), {\em JDK} and Kernel }]
\externalfigure[jbb.unfold.svg][fullwidth]
\stopplacefigure
