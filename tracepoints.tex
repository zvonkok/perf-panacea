\chapter{Tracepoints}
\section{What is a Tracepoint}
A tracepoint provides a hook to call a function (probe) everytime a
tracepoint is encountered. This probes can be supplied by e.g.  {\em perf}
counting or recording variables associated to a specific tracepoint or
can be added programmatically at run time.

/Tracepoints/ can be /on/ (probe attached) or /off/ (probe not
attached).  When /on/ beware that probes are run everytime a
tracepoint is encountered and executed in the context of the
caller. If the probe contains heavy computation the execution of the
traced binary will be slowed down.

When /off/ there is a tiny time penalty (check a condition for a branch)
\starttyping
if (tracepoint_on) { tracepoint_probe(); }
\stoptyping
and a minor space penalty for a auxiliary data structure.

For /Tracepoints/ to work properly with {\em perf} one needs to install the
debug symbols for the to examined binary.
