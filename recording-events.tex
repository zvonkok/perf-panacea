\chapter{sampling events}
\section{perf record}
The second modus operandi is recording (sampling) events. The default frequency of {\em perf}
for sampling is 1000Hz. To sample at a lower frequency just append the flag /-F <freq>/ to
/perf record/. So e.g. to sample the cpu-cycles at 49 Hz for a simple workload one can use
it as follows.
\starttyping
# perf record -F 49  -e cpu-cycles -o ./perf.data -- md5sum /boot/vmlin* | cut -c 1-70
3e9fe83e5d7ecd6cea95928ae5797c4a  /boot/vmlinux-4.11.0-20170505.0.8571
5537e1d73f7e6bcfccf3340f2f0100bf  /boot/vmlinux-4.11.0-20170505.0.8571
dabc2613aaffe1ffc6eb1ae9a1a80f89  /boot/vmlinux-4.11.0-20170505.0.8571
6f6593f862c4a57a64dfb4880196bb99  /boot/vmlinux-4.11.0-20170505.0.daa6
9bd04638d9da0e44dfade5cc70af9817  /boot/vmlinuz
81cad85305e6d4f0a3c71db200edc748  /boot/vmlinuz-4.11.0-00002-g6dc0234-
81cad85305e6d4f0a3c71db200edc748  /boot/vmlinuz-4.11.0-00002-g6dc0234-
668db32e445a13f5adb9e958b02eb56a  /boot/vmlinuz-4.11.0-03422-g028fd7c-
668db32e445a13f5adb9e958b02eb56a  /boot/vmlinuz-4.11.0-03422-g028fd7c-
9bd04638d9da0e44dfade5cc70af9817  /boot/vmlinuz-4.11.0-03424-g99ce1a5-
9bd04638d9da0e44dfade5cc70af9817  /boot/vmlinuz-4.11.0-03424-g99ce1a5-
68b317e587506c383d94be767139c3fe  /boot/vmlinuz-4.11.0-20170505.0.daa6
5e11628b5ae8e092ebca56a2c3bba438  /boot/vmlinuz-4.11.0-rc6-00001-ged58
5e11628b5ae8e092ebca56a2c3bba438  /boot/vmlinuz-4.11.0-rc6-00001-ged58
5b502a065a7a7429a8e540d7759366e7  /boot/vmlinuz-4.12.0-rc4-00524-g1b35
1cb87bbf5e900fca3c61f9311ac7c066  /boot/vmlinuz-4.12.0-rc7
c4f2c214bf421ed0462c0620f808ed49  /boot/vmlinuz-4.12.0-rc7perf
c4f2c214bf421ed0462c0620f808ed49  /boot/vmlinuz-4.12.0-rc7perf.old
\stoptyping

The reason for using 49 as a frequency for sampling is to avoid
sampling in lockstep with other activities that can lead to misleading
results.

After sampling the workload there are several ways how to examine the
gathered data.  The first command is /report/ that shows where the
events are spend in which function. Here as expected the cycles are
spent in the function /md5sum/ from /libcryto.so/.

\starttyping[]
# perf report --stdio --header
# ========
# captured on: Mon Jul 10 12:21:44 2017
# hostname : s311lp09
# os release : 4.11.0-03424-g99ce1a5-dirty
# perf version : 4.11.gead9ae
# arch : s390x
# nrcpus online : 16
# nrcpus avail : 16
# cpudesc : IBM/S390
# cpuid : IBM/S390
# total memory : 264212152 kB
# cmdline : /mnt/6ddc/kernel/latest/tools/perf record -F 49 -e cpu-cycles -- md5sum /boot/vmlinux-4.11.0-20170505.0.8571fc5.6b03710.fc25.s390xdefault /boot/vmlinux-4.11.0-20170505.0.8571fc5.6b03710.fc25.s390xperformance /boot/vmlinux-4.11.0-20170505.0.8571fc5.6b03710.fc25.s390xzfcpdump /boot/vmlinux-4.11.0-20170505.0.daa6a8b.6b03710.fc25.s390xkvm /boot/vmlinuz /boot/vmlinuz-4.11.0-00002-g6dc0234-dirty /boot/vmlinuz-4.11.0-00002-g6dc0234-dirty.old /boot/vmlinuz-4.11.0-03422-g028fd7c-dirty /boot/vmlinuz-4.11.0-03422-g028fd7c-dirty.old /boot/vmlinuz-4.11.0-03424-g99ce1a5-dirty /boot/vmlinuz-4.11.0-03424-g99ce1a5-dirty.old /boot/vmlinuz-4.11.0-20170505.0.daa6a8b.6b03710.fc25.s390xkvm /boot/vmlinuz-4.11.0-rc6-00001-ged585a7-dirty /boot/vmlinuz-4.11.0-rc6-00001-ged585a7-dirty.old /boot/vmlinuz-4.12.0-rc4-00524-g1b35c0a-dirty /boot/vmlinuz-4.12.0-rc7 /boot/vmlinuz-4.12.0-rc7perf /boot/vmlinuz-4.12.0-rc7perf.old
# event : name = cpu-cycles, , size = 112, { sample_period, sample_freq } = 49, sample_type = IP|TID|TIME|PERIOD, disabled = 1, inherit = 1, mmap = 1, comm = 1, freq = 1, enable_on_exec = 1, task = 1, sample_id_all = 1, exclude_guest = 1, mmap2 = 1, comm_exec = 1
# HEADER_CPU_TOPOLOGY info available, use -I to display
# HEADER_NUMA_TOPOLOGY info available, use -I to display
# pmu mappings: cpum_sf = 4, cpum_cf = 4, software = 1, tracepoint = 2
# HEADER_CACHE info available, use -I to display
# missing features: HEADER_TRACING_DATA HEADER_BRANCH_STACK HEADER_GROUP_DESC HEADER_AUXTRACE HEADER_STAT
# ========
#
#
# Total Lost Samples: 0
#
# Samples: 3  of event 'cpu-cycles'
# Event count (approx.): 306120000
#
# Overhead  Command  Shared Object        Symbol
# ........  .......  ...................  ........................
#
   100.00%  md5sum   libcrypto.so.1.1.0e  [.] md5_block_data_order


#
# (Tip: Show user configuration overrides: perf config --user --list)
#
\stoptyping

The second way is to use /perf script/ that shows a more detailed look of
the samples. This mode is more of use when developing new functionality for
perf and debugging perf itself. The section /Heatmap/ will show the use of
/perf script/ to develop a visualization for latency examination.

\starttyping
# perf script -D

0xe8 [0x50]: event: 1
.
. ... raw event: size 80 bytes
.  0000:  00 00 00 01 00 01 00 50 ff ff ff ff 00 00 00 00  .......P........
.  0010:  00 00 00 00 00 00 02 00 00 00 03 ff 80 00 26 70  ..............&p
.  0020:  00 00 00 00 00 00 02 00 5b 6b 65 72 6e 65 6c 2e  ........[kernel.
.  0030:  6b 61 6c 6c 73 79 6d 73 5d 5f 74 65 78 74 00 00  kallsyms]_text..
.  0040:  00 00 00 00 00 00 00 00 00 00 00 00 00 00 00 00  ................

0 0xe8 [0x50]: PERF_RECORD_MMAP -1/0: [0x200(0x3ff80002670) @ 0x200]: x [kernel.kallsyms]_text

0x138 [0x80]: event: 1
.
. ... raw event: size 128 bytes
.  0000:  00 00 00 01 00 01 00 80 ff ff ff ff 00 00 00 00  ................
.  0010:  00 00 03 ff 80 00 28 70 00 00 00 00 00 00 b8 98  ......(p........
.  0020:  00 00 00 00 00 00 00 00 2f 6c 69 62 2f 6d 6f 64  ......../lib/mod
.  0030:  75 6c 65 73 2f 34 2e 31 31 2e 30 2d 30 33 34 32  ules/4.11.0-0342
.  0040:  34 2d 67 39 39 63 65 31 61 35 2d 64 69 72 74 79  4-g99ce1a5-dirty
\stoptyping

The last but not least examination method of /perf.data/ is to
annotate code with the samples gathered. The source code is
interweaved, line-wise with the samples recorded before, here
/cpu-cycles/.

\starttyping
# perf annotate --stdio
 Percent |	Source code & Disassembly of libcrypto.so.1.1.0e for cpu-cy
----------------------------------------------------------------------
         :
         :
         :
         :	Disassembly of section .text:
         :
         :	0000000000113a48 <MD5_Init@@OPENSSL_1_1_0+0x38>:
         :	md5_block_data_order():
         :	#ifndef md5_block_data_order
         :	# ifdef X
         :	#  undef X
         :	# endif
         :	void md5_block_data_order(MD5_CTX *c, const void *data_, si
         :	{
    0.00 :	  113a48:       stmg    %r6,%r15,48(%r15)
    0.00 :	  113a4e:       lay     %r15,-248(%r15)
    0.00 :	  113a54:       stg     %r2,240(%r15)
    0.00 :	  113a5a:       lgr     %r1,%r2
         :	# else
         :	    MD5_LONG XX[MD5_LBLOCK];
         :	#  define X(i)   XX[i]
         :	# endif
         :
         :	    A = c->A;
    0.00 :	  113a5e:       l       %r2,0(%r2)
         :	    B = c->B;
    0.00 :	  113a62:       l       %r11,4(%r1)
         :	    C = c->C;
    0.00 :	  113a66:       l       %r9,8(%r1)
\stoptyping
