* Trace (strace)

Based on the /perf trace/ command it will be shown how to use counting
and recording to examine an application.

By means of a simple /benchmark/, here /dd/ issuing some reads and
writes, it will be shown that {\em perf} is significantly faster than its
counterpart /strace/.

\starttyping
# sample command to compare strace with perf trace performance
# dd if=/dev/zero of=/dev/null bs=512 count=1000k 2>&1
1024000+0 records in
1024000+0 records out
524288000 bytes (524 MB, 500 MiB) copied, 0.290385 s, 1.8 GB/s
\stoptyping

Now lets have a look how long does /strace/ take to count
the syscalls issued in the case of /dd/.

\starttyping
# strace counting of syscalls of dd
strace -c dd if=/dev/zero of=/dev/null bs=512 count=1000k 2>&1
1024000+0 records in
1024000+0 records out
524288000 bytes (524 MB, 500 MiB) copied, 33.4576 s, 15.7 MB/s
% time     seconds  usecs/call     calls    errors syscall
------ ----------- ----------- --------- --------- ----------------
 50.95    1.476851           1   1024003           read
 49.05    1.421691           1   1024003           write
  0.00    0.000048          48         1           execve
  0.00    0.000035           3        14         7 open
  0.00    0.000021           2        10           mmap
  0.00    0.000015           2        10           close
  0.00    0.000009           2         5           fstat
  0.00    0.000009           3         3           mprotect
  0.00    0.000006           2         3         3 access
  0.00    0.000006           2         3           brk
  0.00    0.000004           2         2           dup2
  0.00    0.000004           1         3           rt_sigaction
  0.00    0.000003           3         1           munmap
  0.00    0.000001           1         1           lseek
------ ----------- ----------- --------- --------- ----------------
100.00    2.898703               2048062        10 total
\stoptyping

The running time of the native /dd/ was ~0.3 [s], where the counting
of syscalls with /strace/ took 33.5 [s] (111 times slower). Now lets
see how {\em perf} performs.

\starttyping
# pref  counting of syscalls of dd
perf stat -e 'syscalls:sys_enter_*' dd if=/dev/zero of=/dev/null bs=512 count=1000k 2>&1 | head -n 20
1024000+0 records in
1024000+0 records out
524288000 bytes (524 MB, 500 MiB) copied, 0.39246 s, 1.3 GB/s

 Performance counter stats for 'dd if=/dev/zero of=/dev/null bs=512 count=1000k':

         1,024,003      syscalls:sys_enter_read
         1,024,003      syscalls:sys_enter_write
                 0      syscalls:sys_enter_socket
                 0      syscalls:sys_enter_socketpair
                 0      syscalls:sys_enter_bind
                 0      syscalls:sys_enter_listen
                 0      syscalls:sys_enter_accept4
                 0      syscalls:sys_enter_connect
                 0      syscalls:sys_enter_getsockname
                 0      syscalls:sys_enter_getpeername
                 0      syscalls:sys_enter_sendto
                 0      syscalls:sys_enter_recvfrom
                 0      syscalls:sys_enter_setsockopt
                 0      syscalls:sys_enter_getsockopt
                 0      syscalls:sys_enter_shutdown
                 0      syscalls:sys_enter_sendmsg
\stoptyping

The count of calls for read/write are the same of course but the time
of counting these events took for {\em perf} only 0.4 [s] compared to the
33.5 [s] of strace.

Now record the /trace/ events and examine them with /perf report/. The
samples are saved in a file called /perf.data/, in the directory where
the {\em perf} command was executed. There are events that generate a lot of
data so beware to have enough memory in the case of /ramfs/ as decribed
before or disk space.

\starttyping
# record trace
# perf trace record -a dd if=/dev/zero of=/dev/null bs=512 count=1000k 2>&1
1024000+0 records in
1024000+0 records out
524288000 bytes (524 MB, 500 MiB) copied, 0.963196 s, 544 MB/s
[ perf record: Woken up 0 times to write data ]
Warning:
Processed 4911336 events and lost 5 chunks!

Check IO/CPU overload!

Warning:
50 out of order events recorded.
[ perf record: Captured and wrote 487.895 MB perf.data (4722121 samples) ]
\stoptyping

Let's have a look at the sample data with /perf report/. There are far
more options to /perf report/ than shown in this document so it is
advisable to look at the man pages for each {\em perf} command.


\starttyping
# perf report  --stdio
# To display the perf.data header info, please use --header/--header-only options.
#
#
# Total Lost Samples: 0
#
# Samples: 2M of event 'raw_syscalls:sys_enter'
# Event count (approx.): 2361063
#
# Overhead  Trace output
# ........  .....................................................................................
#
    38.05%  NR 4 (1, 2aa2fd6f000, 200, 2aa2fd6f000, 0, 3ff82376690)
     0.00%  NR 4 (3, 2aa26d5cbc8, 8, 0, 2aa26d5cbc8, 8)
     0.00%  NR 4 (3, 3ff9baad9f0, 870, 1ac9f0, 3ff9baad9f0, 870)
     0.00%  NR 4 (3, 3ff9b907f40, 5e8, 6f40, 3ff9b907f40, 5e8)
     0.00%  NR 4 (3, 3ff9b918500, 5e8, 17500, 3ff9b918500, 5e8)
     0.00%  NR 4 (3, 3ff9b91d348, 870, 1c348, 3ff9b91d348, 870)
     0.00%  NR 4 (3, 3ff9b94abc8, 870, 49bc8, 3ff9b94abc8, 870)
     0.00%  NR 4 (3, 3ff9b94e9e0, 870, 4d9e0, 3ff9b94e9e0, 870)
     0.00%  NR 4 (3, 3ff9b95fdb8, 870, 5edb8, 3ff9b95fdb8, 870)

                               ...

\stoptyping

Each syscall has a unique id (see syscall.h) and the coresponding
syscall for NR 4 is /SYS_write/ with the arguments /(1, 2aa2fd6f000,
200, 2aa2fd6f000, 0, 3ff82376690)/ supplied to the function call.
If one would look at the complete trace, there would be other
syscalls but the two most important in this trace are NR 3,4,
respectively /SYS_read/, /SYS_write/.
